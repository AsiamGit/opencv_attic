\cvclass{DynamicDetector}
An adaptively adjusting detector that iteratively detects until the desired number
of features are found.

Adapters can easily be implemented for any detector through the creation of an Adjuster
object.

Beware that this is not thread safe - as the adjustment of parameters breaks the const
of the detection routine...

\begin{lstlisting}
//sample usage:
//will create a detector that attempts to find 100 - 110 FAST Keypoints, and will at most run
//FAST feature detection 10 times until that number of keypoints are found
Ptr<FeatureDetector> detector(new DynamicDetector (100, 110, 10,new FastAdjuster(20,true)));
 
class CV_EXPORTS DynamicDetector: public FeatureDetector {
public:

	/**min_features the minimum desired features
	 * max_features the maximum desired number of features
	 * max_iters the maximum number of times to try to adjust the feature detector params
	 * 			for the FastAdjuster this can be high, but with Star or Surf this can get time consuming
	 * a an AdjusterAdapter that will do the detection and parameter adjustment
	 */
	DynamicDetector(int min_features, int max_features, int max_iters,
			const Ptr<AdjusterAdapter>& a);
 	...
};
 \end{lstlisting}
\cvclass{AdjusterAdapter}
A feature detector parameter adjuster interface, this is used by the \cvCppCross{DynamicDetector}
and is a wrapper for  \cvCppCross{FeatureDetecto}r that allow them to be adjusted after a detection.

See  \cvCppCross{FastAdjuster}, \cvCppCross{StarAdjuster}, \cvCppCross{SurfAdjuster} for concrete implementations.
\begin{lstlisting}
class AdjusterAdapter: public FeatureDetector {
public:
	/** pure virtual interface
	 */
	virtual ~AdjusterAdapter() {
	}
	/** too few features were detected so, adjust the detector params accordingly
	 * \param min the minimum number of desired features
	 * \param n_detected the number previously detected
	 */
	virtual void tooFew(int min, int n_detected) = 0;
	/** too many features were detected so, adjust the detector params accordingly
	 * \param max the maximum number of desired features
	 * \param n_detected the number previously detected
	 */
	virtual void tooMany(int max, int n_detected) = 0;
	/** are params maxed out or still valid?
	 * \return false if the parameters can't be adjusted any more
	 */
	virtual bool good() const = 0;
};
\end{lstlisting}
 
\cvclass{FastAdjuster}
An \cvCppCross{AdjusterAdapter} for the \cvCppCross{FastFeatureDetector}. This will basically decrement or increment the
threshhold by 1

\begin{lstlisting}
class FastAdjuster FastAdjuster: public AdjusterAdapter {
public:
	/**\param init_thresh the initial threshhold to start with, default = 20
	 * \param nonmax whether to use non max or not for fast feature detection
	 */
	FastAdjuster(int init_thresh = 20, bool nonmax = true);
	...
};
\end{lstlisting}

\cvclass{StarAdjuster}
An \cvCppCross{AdjusterAdapter} for the \cvCppCross{StarFeatureDetector}.  This adjusts the responseThreshhold of 
StarFeatureDetector.
\begin{lstlisting}
class StarAdjuster: public AdjusterAdapter {
	StarAdjuster(double initial_thresh = 30.0);
	...
};
\end{lstlisting}


\cvclass{SurfAdjuster}
An \cvCppCross{AdjusterAdapter} for the \cvCppCross{SurfFeatureDetector}.  This adjusts the responseThreshhold of 
SurfFeatureDetector.
\begin{lstlisting}
class SurfAdjuster: public SurfAdjuster {
	SurfAdjuster();
	...
};
\end{lstlisting}
