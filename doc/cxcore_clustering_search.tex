\section{Clustering and Search in Multi-Dimensional Spaces}

\ifCPy

\cvCPyFunc{KMeans2}
Splits set of vectors by a given number of clusters.

\cvdefC{int cvKMeans2(const CvArr* samples, int nclusters,\par
                      CvArr* labels, CvTermCriteria termcrit,\par
                      int attempts=1, CvRNG* rng=0, \par
                      int flags=0, CvArr* centers=0,\par
                      double* compactness=0);}
\cvdefPy{KMeans2(samples,nclusters,labels,termcrit)-> None}

\begin{description}
\cvarg{samples}{Floating-point matrix of input samples, one row per sample}
\cvarg{nclusters}{Number of clusters to split the set by}
\cvarg{labels}{Output integer vector storing cluster indices for every sample}
\cvarg{termcrit}{Specifies maximum number of iterations and/or accuracy (distance the centers can move by between subsequent iterations)}
\ifC
\cvarg{attempts}{How many times the algorithm is executed using different initial labelings. The algorithm returns labels that yield the best compactness (see the last function parameter)}
\cvarg{rng}{Optional external random number generator; can be used to fully control the function behaviour}
\cvarg{flags}{Can be 0 or \texttt{CV\_KMEANS\_USE\_INITIAL\_LABELS}. The latter
value means that during the first (and possibly the only) attempt, the
function uses the user-supplied labels as the initial approximation
instead of generating random labels. For the second and further attempts,
the function will use randomly generated labels in any case}
\cvarg{centers}{The optional output array of the cluster centers}
\cvarg{compactness}{The optional output parameter, which is computed as
$\sum_i ||\texttt{samples}_i - \texttt{centers}_{\texttt{labels}_i}||^2$
after every attempt; the best (minimum) value is chosen and the
corresponding labels are returned by the function. Basically, the
user can use only the core of the function, set the number of
attempts to 1, initialize labels each time using a custom algorithm
(\texttt{flags=CV\_KMEAN\_USE\_INITIAL\_LABELS}) and, based on the output compactness
or any other criteria, choose the best clustering.}
\fi
\end{description}

The function \texttt{cvKMeans2} implements a k-means algorithm that finds the
centers of \texttt{nclusters} clusters and groups the input samples
around the clusters. On output, $\texttt{labels}_i$ contains a cluster index for
samples stored in the i-th row of the \texttt{samples} matrix.

\ifC
% Example: Clustering random samples of multi-gaussian distribution with k-means
\begin{lstlisting}
#include "cxcore.h"
#include "highgui.h"

void main( int argc, char** argv )
{
    #define MAX_CLUSTERS 5
    CvScalar color_tab[MAX_CLUSTERS];
    IplImage* img = cvCreateImage( cvSize( 500, 500 ), 8, 3 );
    CvRNG rng = cvRNG(0xffffffff);

    color_tab[0] = CV_RGB(255,0,0);
    color_tab[1] = CV_RGB(0,255,0);
    color_tab[2] = CV_RGB(100,100,255);
    color_tab[3] = CV_RGB(255,0,255);
    color_tab[4] = CV_RGB(255,255,0);

    cvNamedWindow( "clusters", 1 );

    for(;;)
    {
        int k, cluster_count = cvRandInt(&rng)%MAX_CLUSTERS + 1;
        int i, sample_count = cvRandInt(&rng)%1000 + 1;
        CvMat* points = cvCreateMat( sample_count, 1, CV_32FC2 );
        CvMat* clusters = cvCreateMat( sample_count, 1, CV_32SC1 );

        /* generate random sample from multigaussian distribution */
        for( k = 0; k < cluster_count; k++ )
        {
            CvPoint center;
            CvMat point_chunk;
            center.x = cvRandInt(&rng)%img->width;
            center.y = cvRandInt(&rng)%img->height;
            cvGetRows( points,
                       &point_chunk,
                       k*sample_count/cluster_count,
                       (k == (cluster_count - 1)) ?
                           sample_count :
                           (k+1)*sample_count/cluster_count );
            cvRandArr( &rng, &point_chunk, CV_RAND_NORMAL,
                       cvScalar(center.x,center.y,0,0),
                       cvScalar(img->width/6, img->height/6,0,0) );
        }

        /* shuffle samples */
        for( i = 0; i < sample_count/2; i++ )
        {
            CvPoint2D32f* pt1 =
                (CvPoint2D32f*)points->data.fl + cvRandInt(&rng)%sample_count;
            CvPoint2D32f* pt2 =
                (CvPoint2D32f*)points->data.fl + cvRandInt(&rng)%sample_count;
            CvPoint2D32f temp;
            CV_SWAP( *pt1, *pt2, temp );
        }

        cvKMeans2( points, cluster_count, clusters,
                   cvTermCriteria( CV_TERMCRIT_EPS+CV_TERMCRIT_ITER, 10, 1.0 ));

        cvZero( img );

        for( i = 0; i < sample_count; i++ )
        {
            CvPoint2D32f pt = ((CvPoint2D32f*)points->data.fl)[i];
            int cluster_idx = clusters->data.i[i];
            cvCircle( img,
                      cvPointFrom32f(pt),
                      2,
                      color_tab[cluster_idx],
                      CV_FILLED );
        }

        cvReleaseMat( &points );
        cvReleaseMat( &clusters );

        cvShowImage( "clusters", img );

        int key = cvWaitKey(0);
        if( key == 27 )
            break;
    }
}
\end{lstlisting}

\cvCPyFunc{SeqPartition}
Splits a sequence into equivalency classes.

\begin{lstlisting}
typedef int (CV_CDECL* CvCmpFunc)(const void* a, const void* b, void* userdata);
\end{lstlisting}

\cvdefC{
int cvSeqPartition( \par const CvSeq* seq,\par CvMemStorage* storage,\par CvSeq** labels,\par CvCmpFunc is\_equal,\par void* userdata );
}

\begin{description}
\cvarg{seq}{The sequence to partition}
\cvarg{storage}{The storage block to store the sequence of equivalency classes. If it is NULL, the function uses \texttt{seq->storage} for output labels}
\cvarg{labels}{Ouput parameter. Double pointer to the sequence of 0-based labels of input sequence elements}
\cvarg{is\_equal}{The relation function that should return non-zero if the two particular sequence elements are from the same class, and zero otherwise. The partitioning algorithm uses transitive closure of the relation function as an equivalency critria}
\cvarg{userdata}{Pointer that is transparently passed to the \texttt{is\_equal} function}
\end{description}

The function \texttt{cvSeqPartition} implements a quadratic algorithm for
splitting a set into one or more equivalancy classes. The function
returns the number of equivalency classes.

% Example: Partitioning a 2d point set
\begin{lstlisting}

#include "cxcore.h"
#include "highgui.h"
#include <stdio.h>

CvSeq* point_seq = 0;
IplImage* canvas = 0;
CvScalar* colors = 0;
int pos = 10;

int is_equal( const void* _a, const void* _b, void* userdata )
{
    CvPoint a = *(const CvPoint*)_a;
    CvPoint b = *(const CvPoint*)_b;
    double threshold = *(double*)userdata;
    return (double)((a.x - b.x)*(a.x - b.x) + (a.y - b.y)*(a.y - b.y)) <=
        threshold;
}

void on_track( int pos )
{
    CvSeq* labels = 0;
    double threshold = pos*pos;
    int i, class_count = cvSeqPartition( point_seq,
                                         0,
                                         &labels,
                                         is_equal,
                                         &threshold );
    printf("%4d classes\n", class_count );
    cvZero( canvas );

    for( i = 0; i < labels->total; i++ )
    {
        CvPoint pt = *(CvPoint*)cvGetSeqElem( point_seq, i );
        CvScalar color = colors[*(int*)cvGetSeqElem( labels, i )];
        cvCircle( canvas, pt, 1, color, -1 );
    }

    cvShowImage( "points", canvas );
}

int main( int argc, char** argv )
{
    CvMemStorage* storage = cvCreateMemStorage(0);
    point_seq = cvCreateSeq( CV_32SC2,
                             sizeof(CvSeq),
                             sizeof(CvPoint),
                             storage );
    CvRNG rng = cvRNG(0xffffffff);

    int width = 500, height = 500;
    int i, count = 1000;
    canvas = cvCreateImage( cvSize(width,height), 8, 3 );

    colors = (CvScalar*)cvAlloc( count*sizeof(colors[0]) );
    for( i = 0; i < count; i++ )
    {
        CvPoint pt;
        int icolor;
        pt.x = cvRandInt( &rng ) % width;
        pt.y = cvRandInt( &rng ) % height;
        cvSeqPush( point_seq, &pt );
        icolor = cvRandInt( &rng ) | 0x00404040;
        colors[i] = CV_RGB(icolor & 255,
                           (icolor >> 8)&255,
                           (icolor >> 16)&255);
    }

    cvNamedWindow( "points", 1 );
    cvCreateTrackbar( "threshold", "points", &pos, 50, on_track );
    on_track(pos);
    cvWaitKey(0);
    return 0;
}
\end{lstlisting}

\fi

\fi

\ifCpp

\cvCppFunc{kmeans}
Finds the centers of clusters and groups the input samples around the clusters.
\cvdefCpp{double kmeans( const Mat\& samples, int clusterCount, Mat\& labels,\par
               TermCriteria termcrit, int attempts,\par
               int flags, Mat* centers );}
\begin{description}
\cvarg{samples}{Floating-point matrix of input samples, one row per sample}
\cvarg{clusterCount}{The number of clusters to split the set by}
\cvarg{labels}{The input/output integer array that will store the cluster indices for every sample}
\cvarg{termcrit}{Specifies maximum number of iterations and/or accuracy (distance the centers can move by between subsequent iterations)}

\cvarg{attempts}{How many times the algorithm is executed using different initial labelings. The algorithm returns the labels that yield the best compactness (see the last function parameter)}
\cvarg{flags}{It can take the following values:
\begin{description}
\cvarg{KMEANS\_RANDOM\_CENTERS}{Random initial centers are selected in each attempt}
\cvarg{KMEANS\_PP\_CENTERS}{Use kmeans++ center initialization by Arthur and Vassilvitskii}
\cvarg{KMEANS\_USE\_INITIAL\_LABELS}{During the first (and possibly the only) attempt, the
function uses the user-supplied labels instaed of computing them from the initial centers. For the second and further attempts, the function will use the random or semi-random centers (use one of \texttt{KMEANS\_*\_CENTERS} flag to specify the exact method)}
\end{description}}
\cvarg{centers}{The output matrix of the cluster centers, one row per each cluster center}
\end{description}

The function \texttt{kmeans} implements a k-means algorithm that finds the
centers of \texttt{clusterCount} clusters and groups the input samples
around the clusters. On output, $\texttt{labels}_i$ contains a 0-based cluster index for
the sample stored in the $i^{th}$ row of the \texttt{samples} matrix.

The function returns the compactness measure, which is computed as
\[
\sum_i \|\texttt{samples}_i - \texttt{centers}_{\texttt{labels}_i}\|^2
\]
after every attempt; the best (minimum) value is chosen and the
corresponding labels and the compactness value are returned by the function.
Basically, the user can use only the core of the function, set the number of
attempts to 1, initialize labels each time using some custom algorithm and pass them with
\par (\texttt{flags}=\texttt{KMEANS\_USE\_INITIAL\_LABELS}) flag, and then choose the best (most-compact) clustering.

\cvCppFunc{partition}
Splits an element set into equivalency classes.

\cvdefCpp{template<typename \_Tp, class \_EqPredicate> int\newline
    partition( const vector<\_Tp>\& vec, vector<int>\& labels,\par
               \_EqPredicate predicate=\_EqPredicate());}
\begin{description}
\cvarg{vec}{The set of elements stored as a vector}
\cvarg{labels}{The output vector of labels; will contain as many elements as \texttt{vec}. Each label \texttt{labels[i]} is 0-based cluster index of \texttt{vec[i]}}
\cvarg{predicate}{The equivalence predicate (i.e. pointer to a boolean function of two arguments or an instance of the class that has the method \texttt{bool operator()(const \_Tp\& a, const \_Tp\& b)}. The predicate returns true when the elements are certainly if the same class, and false if they may or may not be in the same class}
\end{description}

The generic function \texttt{partition} implements an $O(N^2)$ algorithm for
splitting a set of $N$ elements into one or more equivalency classes, as described in \url{http://en.wikipedia.org/wiki/Disjoint-set_data_structure}. The function
returns the number of equivalency classes.

\subsection{Fast Approximate Nearest Neighbor Search}

\def\urltilda{\kern -.05em\lower .7ex\hbox{\~{}}\kern .04em}

This section documents OpenCV's interface to the FLANN\footnote{http://people.cs.ubc.ca/\urltilda mariusm/flann} library. FLANN (Fast Library for Approximate Nearest Neighbors) is a library that
contains a collection of algorithms optimized for fast nearest neighbor search in large datasets and for high dimensional features. More 
information about FLANN can be found in \cite{muja_flann_2009}.

\cvclass{flann::Index}
The FLANN nearest neighbor index class.

\begin{lstlisting}
namespace flann
{
    class Index 
    {
    public:
	    Index(const Mat& features, const IndexParams& params);

	    void knnSearch(const vector<float>& query, 
			   vector<int>& indices, 
			   vector<float>& dists, 
			   int knn, 
			   const SearchParams& params);
	    void knnSearch(const Mat& queries, 
                           Mat& indices, 
                           Mat& dists, 
                           int knn, 
		           const SearchParams& params);

	    int radiusSearch(const vector<float>& query, 
			     vector<int>& indices, 
			     vector<float>& dists, 
			     float radius, 
			     const SearchParams& params);
	    int radiusSearch(const Mat& query, 
			     Mat& indices, 
			     Mat& dists, 
			     float radius, 
			     const SearchParams& params);

	    void save(std::string filename);

	    int veclen() const;

	    int size() const;
    };
}
\end{lstlisting}

\cvCppFunc{flann::Index::Index}
Constructs a nearest neighbor search index for a given dataset.

\cvdefCpp{Index::Index(const Mat\& features, const IndexParams\& params);}
\begin{description}
\cvarg{features}{ Matrix of type CV\_32F containing the features(points) to index. The size of the matrix is num\_features x feature\_dimensionality.}
\cvarg{params}{Structure containing the index parameters. The type of index that will be constructed depends on the type of this parameter.
The possible parameter types are:

\begin{description}
 \cvarg{LinearIndexParams}{When passing an object of this type, the index will perform a linear, brute-force search.
   \cvcode{
    struct LinearIndexParams : public IndexParams\newline
  \{\newline
  \};}
  }

\cvarg{KDTreeIndexParams}{When passing an object of this type the index constructed will consist of a set 
of randomized kd-trees which will be searched in parallel.
 \cvcode{
  struct KDTreeIndexParams : public IndexParams\newline
  \{\newline
	KDTreeIndexParams( int trees = 4 );\newline
  \};}
\begin{description}
\cvarg{trees}{The number of parallel kd-trees to use. Good values are in the range [1..16]}
\end{description}
}
\cvarg{KMeansIndexParams}{When passing an object of this type the index constructed will be a hierarchical k-means tree. 
\cvcode{
  struct KMeansIndexParams : public IndexParams\newline
  \{\newline
	    KMeansIndexParams( int branching = 32,\par
			   int iterations = 11,\par
			   flann\_centers\_init\_t centers\_init = CENTERS\_RANDOM,\par
			   float cb\_index = 0.2 );\newline
  \};}
\begin{description}
\cvarg{branching}{ The branching factor to use for the hierarchical k-means tree }
\cvarg{iterations}{ The maximum number of iterations to use in the k-means clustering 
		    stage when building the k-means tree. A value of -1 used here means
		    that the k-means clustering should be iterated until convergence}
\cvarg{centers\_init}{ The algorithm to use for selecting the initial
		  centers when performing a k-means clustering step. The possible values are
		  CENTERS\_RANDOM (picks the initial cluster centers randomly), CENTERS\_GONZALES (picks the
		  initial centers using Gonzales' algorithm) and CENTERS\_KMEANSPP (picks the initial
		centers using the algorithm suggested in \cite{arthur_kmeanspp_2007}) }
\cvarg{cb\_index}{ This parameter (cluster boundary index) influences the
		  way exploration is performed in the hierarchical kmeans tree. When \texttt{cb\_index} is zero
		  the next kmeans domain to be explored is choosen to be the one with the closest center. 
		  A value greater then zero also takes into account the size of the domain.}
\end{description}
}
\cvarg{CompositeIndexParams}{When using a parameters object of this type the index created combines the randomized kd-trees 
	and the hierarchical k-means tree.
\cvcode{
  struct CompositeIndexParams : public IndexParams\newline 
  \{\newline
	    CompositeIndexParams( int trees = 4,\par
			      int branching = 32,\par
			      int iterations = 11,\par
			      flann\_centers\_init\_t centers\_init = CENTERS\_RANDOM,\par 
			      float cb\_index = 0.2 );\newline
  \};}
}
\cvarg{AutotunedIndexParams}{When passing an object of this type the index created is automatically tuned to offer 
the best performance, by choosing the optimal index type (randomized kd-trees, hierarchical kmeans, linear) and parameters for the
dataset provided.
\cvcode{
  struct AutotunedIndexParams : public IndexParams\newline 
  \{\newline
	    AutotunedIndexParams( float target\_precision = 0.9,\par
			      float build\_weight = 0.01,\par
			      float memory\_weight = 0,\par
			      float sample\_fraction = 0.1 );\newline
  \};}
\begin{description}
\cvarg{target\_precision}{ Is a number between 0 and 1 specifying the
percentage of the approximate nearest-neighbor searches that return the
exact nearest-neighbor. Using a higher value for this parameter gives
more accurate results, but the search takes longer. The optimum value
usually depends on the application. }

\cvarg{build\_weight}{ Specifies the importance of the
index build time raported to the nearest-neighbor search time. In some
applications it's acceptable for the index build step to take a long time
if the subsequent searches in the index can be performed very fast. In
other applications it's required that the index be build as fast as
possible even if that leads to slightly longer search times.}

\cvarg{memory\_weight} {Is used to specify the tradeoff between
time (index build time and search time) and memory used by the index. A
value less than 1 gives more importance to the time spent and a value
greater than 1 gives more importance to the memory usage.}

\cvarg{sample\_fraction} {Is a number between 0 and 1 indicating what fraction
of the dataset to use in the automatic parameter configuration algorithm. Running the 
algorithm on the full dataset gives the most accurate results, but for
very large datasets can take longer than desired. In such case using just a fraction of the
data helps speeding up this algorithm while still giving good approximations of the
optimum parameters.}
\end{description}
}
 \cvarg{SavedIndexParams}{This object type is used for loading a previously saved index from the disk.
\cvcode{
  struct SavedIndexParams : public IndexParams\newline
  \{\newline
	   SavedIndexParams( std::string filename );\newline
  \};}
\begin{description}
\cvarg{filename}{ The filename in which the index was saved. }
\end{description}
}
\end{description}
}
\end{description}

\cvCppFunc{flann::Index::knnSearch}
Performs a K-nearest neighbor search for a given query point using the index.
\cvdefCpp{void Index::knnSearch(const vector<float>\& query, \par
		vector<int>\& indices, \par
		vector<float>\& dists, \par
		int knn, \par
		const SearchParams\& params);}
\begin{description}
\cvarg{query}{The query point}
\cvarg{indices}{Vector that will contain the indices of the K-nearest neighbors found. It must have at least knn size.}
\cvarg{dists}{Vector that will contain the distances to the K-nearest neighbors found. It must have at least knn size.}
\cvarg{knn}{Number of nearest neighbors to search for.}
\cvarg{params}{Search parameters}
\begin{lstlisting}
  struct SearchParams {
	  SearchParams(int checks = 32);
  };
\end{lstlisting}
\begin{description}
\cvarg{checks}{ The number of times the tree(s) in the index should be recursively traversed. A
higher value for this parameter would give better search precision, but
also take more time. If automatic configuration was used when the
index was created, the number of checks required to achieve the specified
precision was also computed, in which case this parameter is ignored.}
\end{description}
\end{description}

\cvCppFunc{flann::Index::knnSearch}
Performs a K-nearest neighbor search for multiple query points.

\cvdefCpp{void Index::knnSearch(const Mat\& queries,\par
		Mat\& indices, Mat\& dists,\par
		int knn, const SearchParams\& params);}

\begin{description}
\cvarg{queries}{The query points, one per row}
\cvarg{indices}{Indices of the nearest neighbors found }
\cvarg{dists}{Distances to the nearest neighbors found}
\cvarg{knn}{Number of nearest neighbors to search for}
\cvarg{params}{Search parameters}
\end{description}


\cvCppFunc{flann::Index::radiusSearch}
Performs a radius nearest neighbor search for a given query point.
\cvdefCpp{int Index::radiusSearch(const vector<float>\& query, \par
		  vector<int>\& indices, \par
		  vector<float>\& dists, \par
		  float radius, \par
		  const SearchParams\& params);}
\begin{description}
\cvarg{query}{The query point}
\cvarg{indices}{Vector that will contain the indices of the points found within the search radius in decreasing order of the distance to the query point. If the number of neighbors in the search radius is bigger than the size of this vector, the ones that don't fit in the vector are ignored. }
\cvarg{dists}{Vector that will contain the distances to the points found within the search radius}
\cvarg{radius}{The search radius}
\cvarg{params}{Search parameters}
\end{description}


\cvCppFunc{flann::Index::radiusSearch}
Performs a radius nearest neighbor search for multiple query points.
\cvdefCpp{int Index::radiusSearch(const Mat\& query, \par
		  Mat\& indices, \par
		  Mat\& dists, \par
		  float radius, \par
		  const SearchParams\& params);}
\begin{description}
\cvarg{queries}{The query points, one per row}
\cvarg{indices}{Indices of the nearest neighbors found}
\cvarg{dists}{Distances to the nearest neighbors found}
\cvarg{radius}{The search radius}
\cvarg{params}{Search parameters}
\end{description}


\cvCppFunc{flann::Index::save}
Saves the index to a file.
\cvdefCpp{void Index::save(std::string filename);}
\begin{description}
\cvarg{filename}{The file to save the index to}
\end{description}


\cvCppFunc{flann::hierarchicalClustering}
Clusters the given points by constructing a hierarchical k-means tree and choosing a cut in the tree that minimizes the cluster's variance.
\cvdefCpp{int hierarchicalClustering(const Mat\& features, Mat\& centers,\par
                                      const KMeansIndexParams\& params);}
\begin{description}
\cvarg{features}{The points to be clustered}
\cvarg{centers}{The centers of the clusters obtained. The number of rows in this matrix represents the number of clusters desired, 
however, because of the way the cut in the hierarchical tree is choosen, the number of clusters computed will be
 the highest number of the form $(branching-1)*k+1$ that's lower than the number of clusters desired, where $branching$ is the tree's 
branching factor (see description of the KMeansIndexParams).  }
\cvarg{params}{Parameters used in the construction of the hierarchical k-means tree}
\end{description}
The function returns the number of clusters computed.

\fi
