\section{Initalization and Information}


\cvCppFunc{gpu::getCudaEnabledDeviceCount}
Returns number of CUDA-enabled devices installed. It is to be used before any other GPU functions calls. If OpenCV is compiled without GPU support this function returns 0. 

\cvdefCpp{int getCudaEnabledDeviceCount();}


\cvCppFunc{gpu::setDevice}
Sets device and initializes it for the current thread. Call of this function can be omitted, but in this case a default device will be initialized on fist GPU usage.

\cvdefCpp{void setDevice(int device);}
\begin{description}
\cvarg{device}{index of GPU device in system starting with 0.}
\end{description}


\cvCppFunc{gpu::getDevice}
Returns the current device index, which was set by \cvCppCross{gpu::getDevice} or initialized by default.

\cvdefCpp{int getDevice();}


\cvCppFunc{gpu::getComputeCapability}
Returns compute capability version for the given device.

\cvdefCpp{void getComputeCapability(int device, int\& major, int\& minor);}
\begin{description}
\cvarg{device}{GPU index. Can be obtained via \cvCppCross{gpu::getDevice}.}
\cvarg{major}{Major CC version.}
\cvarg{minor}{Minor CC version.}
\end{description}

\cvCppFunc{gpu::getNumberOfSMs}
Returns number of Streaming Multiprocessors for given device.

\cvdefCpp{int getNumberOfSMs(int device);}
\begin{description}
\cvarg{device}{GPU index. Can be obtained via \cvCppCross{gpu::getDevice}.}
\end{description}


\cvCppFunc{gpu::getGpuMemInfo}
Returns free and total memory size for the current device.

\cvdefCpp{void getGpuMemInfo(size\_t\& free, size\_t\& total);}
\begin{description}
\cvarg{free}{Reference to free GPU memory counter.}
\cvarg{total}{Reference to total GPU memory counter.}
\end{description}


\cvCppFunc{gpu::hasNativeDoubleSupport}
Returns true, if the specified GPU has native double support, otherwise false.

\cvdefCpp{bool hasNativeDoubleSupport(int device);}
\begin{description}
\cvarg{device}{GPU index. Can be obtained via \cvCppCross{gpu::getDevice}.}
\end{description}


\cvCppFunc{gpu::hasAtomicsSupport}
Returns true, if the specified GPU has atomics support, otherwise false.

\cvdefCpp{bool hasAtomicsSupport(int device);}
\begin{description}
\cvarg{device}{GPU index. Can be obtained via \cvCppCross{gpu::getDevice}.}
\end{description} 

\cvclass{gpu::TargetArchs}
This class provides functionality (as set of static methods) for checking which NVIDIA card architectures the GPU module was built for.

\bigskip

The following method checks whether the module was built with the support of the given feature:
\cvdefCpp{static bool builtWith(GpuFeature feature);}
\begin{description}
\cvarg{feature}{Feature to be checked. Available alternatives:
\begin{itemize}
\item NATIVE\_DOUBLE Native double operations support
\item ATOMICS Atomic operations support
\end{itemize}}
\end{description}

There are a set of methods for checking whether the module contains intermediate (PTX) or binary GPU code for the given architecture:
\cvdefCpp{
static bool has(int major, int minor);\newline
static bool hasPtx(int major, int minor);\newline
static bool hasBin(int major, int minor);\newline
static bool hasEqualOrLessPtx(int major, int minor);\newline
static bool hasEqualOrGreater(int major, int minor);\newline
static bool hasEqualOrGreaterPtx(int major, int minor);\newline
static bool hasEqualOrGreaterBin(int major, int minor);}
\begin{description}
\cvarg{major}{Major compute capability version.}
\cvarg{minor}{Minor compute capability version.}
\end{description}

\cvCppFunc{gpu::isCompatibleWith}
Returns true, if the GPU module is built with PTX or CUBIN compatible with the given GPU device, otherwise false.

\cvdefCpp{bool isCompatibleWith(int device);}
\begin{description}
\cvarg{device}{GPU index. Can be obtained via \cvCppCross{gpu::getDevice}.}
\end{description}

% By default GPU module is no compiled for devices with compute capability equal to 1.0. So if you run

According to the CUDA C Programming Guide Version 3.2: "PTX code produced for some specific compute capability can always be compiled to binary code of greater or equal compute capability". 

