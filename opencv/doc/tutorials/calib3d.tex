
%%%%%%%%%%%%%%%%%%%%%%%%%%%%%%%%%%%%%%%%%%%%%%%%%%%%%%%%%%%%%%%%%%%%%%%%%%%%%%%%%%%%%%
%                                                                                    %
%                                        C++                                         %
%                                                                                    %
%%%%%%%%%%%%%%%%%%%%%%%%%%%%%%%%%%%%%%%%%%%%%%%%%%%%%%%%%%%%%%%%%%%%%%%%%%%%%%%%%%%%%%

\ifCpp
\section{Camera calibration}
The goal of this tutorial is to learn how to calibrate a camera given a set of chessboard images. 

\texttt{Test data}: use images in your data/chess folder. 

Compile opencv with samples by setting BUILD\_EXAMPLES to ON in cmake configuration. 

Go to bin folder and use \texttt{imagelist\_creator} to create an xml/yaml list of your images. Then, run \texttt{calibration} sample to get camera parameters. Use square size equal to 3cm. 

\section{Pose estimation}
Now, let us write a code that detects a chessboard in a new image and finds its distance from the camera. You can apply the same method to any object with knwon 3d geometry that you can detect in an image.

\texttt{Test data}: use chess\_test*.jpg images from your data folder.

Create an empty console project. Load a test image:
\begin{lstlisting}
Mat img = imread(argv[1], CV_LOAD_IMAGE_GRAYSCALE);
\end{lstlisting}

Detect a chessboard in this image using findChessboard function. 
\begin{lstlisting}
bool found = findChessboardCorners( img, boardSize, ptvec, CV_CALIB_CB_ADAPTIVE_THRESH );
\end{lstlisting}

Now, write a function that generates a \texttt{vector<Point3f>} array of 3d coordinates of a chessboard in any coordinate system. For simplicity, let us choose a system such that one of the chessboard corners is in the origin and the board is in the plane \(z = 0\).

Read camera parameters from xml/yaml file:
\begin{lstlisting}
FileStorage fs(filename, FileStorage::READ);
Mat intrinsics, distortion;
fs["camera_matrix"] >> intrinsics;
fs["distortion_coefficients"] >> distortion;
\end{lstlisting}

Now we are ready to find chessboard pose by running solvePnP:
\begin{lstlisting}
vector<Point3f> boardPoints;
// fill the array
...

solvePnP(Mat(boardPoints), Mat(foundBoardCorners), cameraMatrix,
                     distCoeffs, rvec, tvec, false);
\end{lstlisting}

Calculate reprojection error like it is done in \texttt{calibration} sample (see textttt{opencv/samples/cpp/calibration.cpp}, function \texttt{computeReprojectionErrors}). 

How to calculate the distance from the camera origin to any of the corners? 
\fi