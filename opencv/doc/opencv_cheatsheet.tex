%
%    The OpenCV cheatsheet structure:
%
%    creating matrices
%        from scratch
%        from previously allocated data: plain arrays, vectors 
%        converting to/from old-style structures
%
%    element access, iteration through matrix elements
%
%    copying & shuffling data
%        copying & converting the whole matrices
%        extracting matrix parts & copying them
%        split, merge & mixchannels
%        flip, transpose, repeat
%
%    matrix & image operations:
%        arithmetics & logic
%        matrix multiplication, inversion, determinant, trace, SVD
%        statistical functions
%
%    basic image processing:
%        image filtering with predefined & custom filters
%        example: finding local maxima
%        geometrical transformations, resize, warpaffine, perspective & remap.
%        color space transformations
%        histograms & back projections
%        contours
%        
%    i/o:
%        displaying images
%        saving/loading to/from file (XML/YAML & image file formats)
%        reading videos & camera feed, writing videos
%
%    operations on point sets:
%        findcontours, bounding box, convex hull, min area rect,
%            transformations, to/from homogeneous coordinates
%        matching point sets: homography, fundamental matrix, rigid transforms
%        
%    3d:
%        camera calibration, pose estimation.
%        uncalibrated case
%        stereo: rectification, running stereo correspondence, obtaining the depth.
%    
%    feature detection:
%        features2d toolbox
%    
%    object detection:
%        using a classifier running on a sliding window: cascadeclassifier + hog.
%        using salient point features: features2d -> matching
%        
%    statistical data processing:
%        clustering (k-means),
%        classification + regression (SVM, boosting, k-nearest),
%        compressing data (PCA)
%

\documentclass[10pt,landscape]{article}
\usepackage[usenames,dvips,pdftex]{color}
\usepackage{multicol}
\usepackage{calc}
\usepackage{ifthen}
\usepackage[pdftex]{color,graphicx}
\usepackage[landscape]{geometry}
\usepackage{hyperref}
\hypersetup{colorlinks=true, filecolor=black, linkcolor=black, urlcolor=blue, citecolor=black}
\graphicspath{{./images/}}


% This sets page margins to .5 inch if using letter paper, and to 1cm
% if using A4 paper. (This probably isn't strictly necessary.)
% If using another size paper, use default 1cm margins.
\ifthenelse{\lengthtest { \paperwidth = 11in}}
	{ \geometry{top=.5in,left=.5in,right=.5in,bottom=.5in} }
	{\ifthenelse{ \lengthtest{ \paperwidth = 297mm}}
		{\geometry{top=1cm,left=1cm,right=1cm,bottom=1cm} }
		{\geometry{top=1cm,left=1cm,right=1cm,bottom=1cm} }
	}

% Turn off header and footer
\pagestyle{empty}

% Redefine section commands to use less space
\makeatletter
\renewcommand{\section}{\@startsection{section}{1}{0mm}%
                                {-1ex plus -.5ex minus -.2ex}%
                                {0.5ex plus .2ex}%x
                                {\normalfont\large\bfseries}}
\renewcommand{\subsection}{\@startsection{subsection}{2}{0mm}%
                                {-1explus -.5ex minus -.2ex}%
                                {0.5ex plus .2ex}%
                                {\normalfont\normalsize\bfseries}}
\renewcommand{\subsubsection}{\@startsection{subsubsection}{3}{0mm}%
                                {-1ex plus -.5ex minus -.2ex}%
                                {1ex plus .2ex}%
                                {\normalfont\small\bfseries}}
\makeatother

% Define BibTeX command
\def\BibTeX{{\rm B\kern-.05em{\sc i\kern-.025em b}\kern-.08em
    T\kern-.1667em\lower.7ex\hbox{E}\kern-.125emX}}

% Don't print section numbers
\setcounter{secnumdepth}{0}


%\setlength{\parindent}{0pt}
%\setlength{\parskip}{0pt plus 0.5ex}

\newcommand{\ccode}[1]{
\begin{alltt}
#1
\end{alltt}
}

% -----------------------------------------------------------------------

\begin{document}

\raggedright
\footnotesize
\begin{multicols}{3}


% multicol parameters
% These lengths are set only within the two main columns
%\setlength{\columnseprule}{0.25pt}
\setlength{\premulticols}{1pt}
\setlength{\postmulticols}{1pt}
\setlength{\multicolsep}{1pt}
\setlength{\columnsep}{2pt}

\begin{center}
     \Large{\textbf{OpenCV 2.1+ Cheat Sheet}} \\
\end{center}
\newlength{\MyLen}
\settowidth{\MyLen}{\texttt{letterpaper}/\texttt{a4paper} \ }

%\section{Filesystem Concepts}
%\begin{tabular}{@{}p{\the\MyLen}%
 %               @{}p{\linewidth-\the\MyLen}@{}}
%\texttt{\href{http://www.ros.org/wiki/Packages}{package}}   & The lowest level of ROS software organization. \\
%\texttt{\href{http://www.ros.org/wiki/Manifest}{manifest}}  & Description of a ROS package. \\
%\texttt{\href{http://www.ros.org/wiki/Stack}{stack}} & Collections of ROS packages that form a higher-level library. \\
%\texttt{\href{http://www.ros.org/wiki/Stack Manifest}{stack manifest}}  & Description of a ROS stack.
%\end{tabular}

\section{Matrix Basics}
\begin{tabbing}

\textbf{Cr}\=\textbf{ea}\=\textbf{te}\={} \textbf{a matrix} \\
\> \texttt{Mat image(240, 320, CV\_8UC3);} \\

\textbf{[Re]allocate a pre-declared matrix}\\
\> \texttt{image.\href{http://opencv.willowgarage.com/documentation/cpp/basic_structures.html\#Mat::create}{create}(480, 640, CV\_8UC3);}\\

\textbf{Create a matrix initialized with a constant}\\
\> \texttt{Mat A33(3, 3, CV\_32F, Scalar(5));} \\
\> \texttt{Mat B33(3, 3, CV\_32F); B33 = Scalar(5);} \\
\> \texttt{Mat C33 = Mat::ones(3, 3, CV\_32F)*5.;} \\
\> \texttt{Mat D33 = Mat::zeros(3, 3, CV\_32F) + 5.;} \\

\textbf{Create a matrix initialized with specified values}\\
\> \texttt{double a = CV\_PI/3;} \\
\> \texttt{Mat A22 = Mat(Mat\_<float>(2, 2) <<} \\
\> \> \texttt{cos(a), -sin(a), sin(a), cos(a));} \\
\> \texttt{float B22data[] = \{cos(a), -sin(a), sin(a), cos(a)\};} \\
\> \texttt{Mat B22 = Mat(2, 2, CV\_32F, B22data).clone();}\\

\textbf{Initialize a random matrix}\\
\> \texttt{\href{http://opencv.willowgarage.com/documentation/cpp/operations_on_arrays.html\#cv-randu}{randu}(image, Scalar(0), Scalar(256)); }\textit{// uniform dist}\\
\> \texttt{\href{http://opencv.willowgarage.com/documentation/cpp/operations_on_arrays.html\#cv-randn}{randn}(image, Scalar(128), Scalar(10)); }\textit{// Gaussian dist}\\

\textbf{Convert matrix to/from other structures}\\
\>(without copying the data)\\
\> \texttt{Mat image\_alias = image;}\\
\> \texttt{float* Idata=new float[480*640*3];}\\
\> \texttt{Mat I(480, 640, CV\_32FC3, Idata);}\\
\> \texttt{vector<Point> iptvec(10);}\\
\> \texttt{Mat iP(iptvec); }\textit{// iP -- 10x1 CV\_32SC2 matrix}\\
\> \texttt{CvMat* oldC0 = cvCreateImage(cvSize(320, 240), 16);}\\
\> \texttt{Mat newC = cvarrToMat(oldC0);}\\
\> \texttt{IplImage oldC1 = newC; CvMat oldC2 = newC;}\\

\textbf{... (with copying the data)}\\
\> \texttt{Mat image\_copy = image.clone();}\\
\> \texttt{Mat P(10, 1, CV\_32FC2, Scalar(1, 1));}\\
\> \texttt{vector<Point2f> ptvec = Mat\_<Point2f>(P);}\\

\>\\
\textbf{Access matrix elements}\\
\> \texttt{A33.at<float>(i,j) = A33.at<float>(j,i)+1;}\\
\> \texttt{Mat dyImage(image.size(), image.type());}\\
\> \texttt{for(int y = 1; y < image.rows-1; y++) \{}\\
\> \> \texttt{Vec3b* prevRow = image.ptr<Vec3b>(y-1);}\\
\> \> \texttt{Vec3b* nextRow = image.ptr<Vec3b>(y+1);}\\
\> \> \texttt{for(int x = 0; y < image.cols; x++)}\\
\> \> \> \texttt{for(int c = 0; c < 3; c++)}\\
\> \> \> \texttt{  dyImage.at<Vec3b>(y,x)[c] =}\\
\> \> \> \texttt{    saturate\_cast<uchar>(}\\
\> \> \> \texttt{       nextRow[x][c] - prevRow[x][c]);}\\
\> \texttt{\} }\\
\> \texttt{Mat\_<Vec3b>::iterator it = image.begin<Vec3b>(),}\\
\> \> \texttt{itEnd = image.end<Vec3b>();}\\
\> \texttt{for(; it != itEnd; ++it)}\\
\> \> \texttt{(*it)[1] \textasciicircum{}= 255;}\\

\end{tabbing}

\section{Matrix Manipulations: Copying, Shuffling, Part Access}
\begin{tabular}{@{}p{\the\MyLen}%
                @{}p{\linewidth-\the\MyLen}@{}}
\texttt{\href{http://opencv.willowgarage.com/documentation/cpp/basic_structures.html\#Mat::copyTo}{Mat::copyTo()}} & Copy matrix to another one \\
\texttt{\href{http://opencv.willowgarage.com/documentation/cpp/basic_structures.html\#Mat::convertTo}{Mat::convertTo()}} & Scale and convert matrix to the specified data type \\
\texttt{\href{http://opencv.willowgarage.com/documentation/cpp/basic_structures.html\#Mat::clone}{Mat::clone()}} & Make deep copy of a matrix \\
\texttt{\href{http://opencv.willowgarage.com/documentation/cpp/basic_structures.html\#Mat::reshape}{Mat::reshape()}} & Change matrix dimensions and/or number of channels without copying data \\

\texttt{\href{http://opencv.willowgarage.com/documentation/cpp/basic_structures.html\#Mat::row}{Mat::row()}}, \texttt{\href{http://opencv.willowgarage.com/documentation/cpp/basic_structures.html\#Mat::rowRange}{Mat::rowRange()}} & Take a matrix row (row span) \\

\texttt{\href{http://opencv.willowgarage.com/documentation/cpp/basic_structures.html\#Mat::col}{Mat::col()}}, \texttt{\href{http://opencv.willowgarage.com/documentation/cpp/basic_structures.html\#Mat::colRange}{Mat::colRange()}} & Take a matrix column (column span) \\

\texttt{\href{http://opencv.willowgarage.com/documentation/cpp/basic_structures.html\#Mat::diag}{Mat::diag()}} & Take a matrix diagonal/create a diagonal matrix \\

\texttt{\href{http://opencv.willowgarage.com/documentation/cpp/basic_structures.html\#index-1245}{Mat::operator ()()}} & Take a submatrix \\

\texttt{\href{http://opencv.willowgarage.com/documentation/cpp/basic_structures.html\#Mat::repeat}{Mat::repeat()}} & Make a bigger matrix by repeating a smaller one \\

\texttt{\href{http://opencv.willowgarage.com/documentation/cpp/operations_on_arrays.html\#cv-flip}{flip()}} & Reverse the order of matrix rows and/or columns \\

\texttt{\href{http://opencv.willowgarage.com/documentation/cpp/operations_on_arrays.html\#cv-split}{split()}} & Split multi-channel matrix into separate channels \\

\texttt{\href{http://opencv.willowgarage.com/documentation/cpp/operations_on_arrays.html\#cv-merge}{merge()}} & Make a multi-channel matrix out of the separate channels \\

\texttt{\href{http://opencv.willowgarage.com/documentation/cpp/operations_on_arrays.html\#cv-mixchannels}{mixChannels()}} & Generalized form of split() and merge() \\

\texttt{\href{http://opencv.willowgarage.com/documentation/cpp/operations_on_arrays.html\#cv-randshuffle}{randShuffle()}} & Randomly shuffle matrix elements \\

\end{tabular}


\section{Simple Matrix Operations}

OpenCV implements most common arithmetical, logical and
other matrix operations, such as

\begin{itemize}
\item
\texttt{\href{http://opencv.willowgarage.com/documentation/cpp/operations_on_arrays.html\#cv-add}{add()}},  \texttt{\href{http://opencv.willowgarage.com/documentation/cpp/operations_on_arrays.html\#cv-subtract}{subtract()}},
\texttt{\href{http://opencv.willowgarage.com/documentation/cpp/operations_on_arrays.html\#cv-multiply}{multiply()}},
\texttt{\href{http://opencv.willowgarage.com/documentation/cpp/operations_on_arrays.html\#cv-divide}{divide()}},
\texttt{\href{http://opencv.willowgarage.com/documentation/cpp/operations_on_arrays.html\#cv-absdiff}{absdiff()}},
\texttt{\href{http://opencv.willowgarage.com/documentation/cpp/operations_on_arrays.html\#bitwise-and}{bitwise\_and()}},  \texttt{\href{http://opencv.willowgarage.com/documentation/cpp/operations_on_arrays.html\#bitwise-or}{bitwise\_or()}},
\texttt{\href{http://opencv.willowgarage.com/documentation/cpp/operations_on_arrays.html\#bitwise-xor}{bitwise\_xor()}},
\texttt{\href{http://opencv.willowgarage.com/documentation/cpp/operations_on_arrays.html\#cv-max}{max()}},
\texttt{\href{http://opencv.willowgarage.com/documentation/cpp/operations_on_arrays.html\#cv-min}{min()}},
\texttt{\href{http://opencv.willowgarage.com/documentation/cpp/operations_on_arrays.html\#cv-compare}{compare()}}

-- correspondingly, addition, subtraction, element-wise multiplication ... comparison of two matrices or a matrix and a scalar.

% (a, a, a, 255)*(r, g, b, a)/255
% 255 - (a, a, a, 255) = (255 - a, ..., 0)
% (b, b, b, b)*(255 - a, 255 - a, 255 - a, 0)/255 = ((255 - a)*b/255, ...., (255 - a))

\begin{tabbing}
Exa\=mple. \href{http://en.wikipedia.org/wiki/Alpha_compositing}{Alpha compositing} function:\\
\texttt{void alphaCompose(const Mat\& rgba1,}\\
\> \texttt{const Mat\& rgba2, Mat\& rgba\_dest)}\\
\texttt{\{ }\\
\> \texttt{Mat a1(rgba1.size(), rgba1.type), ra1;}\\
\> \texttt{Mat a2(rgba2.size(), rgba2.type);}\\
\> \texttt{int mixch[]=\{3, 0, 3, 1, 3, 2, 3, 3\};}\\
\> \texttt{mixChannels(\&rgba1, \&a1, mixch, 4);}\\
\> \texttt{mixChannels(\&rgba2, \&a2, mixch, 4);}\\
\> \texttt{subtract(Scalar::all(255), a1, ra1);}\\
\> \texttt{bitwise\_or(a1, Scalar(0,0,0,255), a1);}\\
\> \texttt{bitwise\_or(a2, Scalar(0,0,0,255), a2);}\\
\> \texttt{multiply(a2, ra1, a2, 1./255);}\\
\> \texttt{multiply(a1, rgba1, a1, 1./255);}\\
\> \texttt{multiply(a2, rgba2, a2, 1./255);}\\
\> \texttt{add(a1, a2, rgba\_dest);}\\
\texttt{\}}
\end{tabbing}

\item

\texttt{\href{http://opencv.willowgarage.com/documentation/cpp/operations_on_arrays.html\#cv-sum}{sum()}},
\texttt{\href{http://opencv.willowgarage.com/documentation/cpp/operations_on_arrays.html\#cv-mean}{mean()}},
\texttt{\href{http://opencv.willowgarage.com/documentation/cpp/operations_on_arrays.html\#cv-mean-stddev}{meanStdDev()}},
\texttt{\href{http://opencv.willowgarage.com/documentation/cpp/operations_on_arrays.html\#cv-norm}{norm()}},
\texttt{\href{http://opencv.willowgarage.com/documentation/cpp/operations_on_arrays.html\#cv-countnonzero}{countNonZero()}},
\texttt{\href{http://opencv.willowgarage.com/documentation/cpp/operations_on_arrays.html\#cv-minmaxloc}{minMaxLoc()}},

-- various statistics of matrix elements.

\item
\texttt{\href{http://opencv.willowgarage.com/documentation/cpp/operations_on_arrays.html\#cv-exp}{exp()}},
\texttt{\href{http://opencv.willowgarage.com/documentation/cpp/operations_on_arrays.html\#cv-log}{log()}},
\texttt{\href{http://opencv.willowgarage.com/documentation/cpp/operations_on_arrays.html\#cv-pow}{pow()}},
\texttt{\href{http://opencv.willowgarage.com/documentation/cpp/operations_on_arrays.html\#cv-sqrt}{sqrt()}},
\texttt{\href{http://opencv.willowgarage.com/documentation/cpp/operations_on_arrays.html\#cv-carttopolar}{cartToPolar()}},
\texttt{\href{http://opencv.willowgarage.com/documentation/cpp/operations_on_arrays.html\#cv-polarToCart}{polarToCart()}}

-- the classical math functions + conversion of Cartesian to polar coordinates and back.

\item
\texttt{\href{http://opencv.willowgarage.com/documentation/cpp/operations_on_arrays.html\#cv-scaleadd}{scaleAdd()}},
\texttt{\href{http://opencv.willowgarage.com/documentation/cpp/operations_on_arrays.html\#cv-transpose}{transpose()}},
\texttt{\href{http://opencv.willowgarage.com/documentation/cpp/operations_on_arrays.html\#cv-gemm}{gemm()}},
\texttt{\href{http://opencv.willowgarage.com/documentation/cpp/operations_on_arrays.html\#cv-invert}{invert()}},
\texttt{\href{http://opencv.willowgarage.com/documentation/cpp/operations_on_arrays.html\#cv-solve}{solve()}},
\texttt{\href{http://opencv.willowgarage.com/documentation/cpp/operations_on_arrays.html\#cv-determinant}{determinant()}},
\texttt{\href{http://opencv.willowgarage.com/documentation/cpp/operations_on_arrays.html\#cv-trace}{trace()}}
\texttt{\href{http://opencv.willowgarage.com/documentation/cpp/operations_on_arrays.html\#cv-eigen}{eigen()}},
\texttt{\href{http://opencv.willowgarage.com/documentation/cpp/operations_on_arrays.html\#cv-SVD}{SVD}},

-- the algebraic functions + SVD class.

\item
\texttt{\href{http://opencv.willowgarage.com/documentation/cpp/operations_on_arrays.html\#cv-dft}{dft()}},
\texttt{\href{http://opencv.willowgarage.com/documentation/cpp/operations_on_arrays.html\#cv-idft}{idft()}},
\texttt{\href{http://opencv.willowgarage.com/documentation/cpp/operations_on_arrays.html\#cv-dct}{dct()}},
\texttt{\href{http://opencv.willowgarage.com/documentation/cpp/operations_on_arrays.html\#cv-idct}{idct()}},

-- discrete Fourier and cosine transformations 

\end{itemize}

For many of the basic operations the alternative algebraic notation can be used, for example:
\begin{tabbing}
\texttt{Mat}\={} \texttt{delta = (J.t()*J + lambda*}\\
\>\texttt{Mat::eye(J.cols, J.cols, J.type())}\\
\>\texttt{.inv(CV\_SVD)*(J.t()*err);}
\end{tabbing}
implements the core of Levenberg-Marquardt optimization algorithm.
Please, see the \href{http://opencv.willowgarage.com/documentation/cpp/basic_structures.html#matrix-expressions}{Matrix Expressions Reference} for details.


\section{Image Processsing}

\subsection{Filtering}

\begin{tabular}{@{}p{\the\MyLen}%
                @{}p{\linewidth-\the\MyLen}@{}}
\texttt{\href{http://opencv.willowgarage.com/documentation/cpp/image_filtering.html\#cv-filter2d}{filter2D()}} & Apply a non-separable linear filter to the image \\

\texttt{\href{http://opencv.willowgarage.com/documentation/cpp/image_filtering.html\#cv-sepfilter2d}{sepFilter2D()}} & Apply a separable linear filter to the image \\

\texttt{\href{http://opencv.willowgarage.com/documentation/cpp/image_filtering.html\#cv-blur}{boxFilter()}},  \texttt{\href{http://opencv.willowgarage.com/documentation/cpp/image_filtering.html\#cv-gaussianblur}{GaussianBlur()}},
\texttt{\href{http://opencv.willowgarage.com/documentation/cpp/image_filtering.html\#cv-medianblur}{medianBlur()}}
& Smooth the image with one of the linear or non-linear filters \\

\texttt{\href{http://opencv.willowgarage.com/documentation/cpp/image_filtering.html\#cv-sobel}{Sobel()}},  \texttt{\href{http://opencv.willowgarage.com/documentation/cpp/image_filtering.html\#cv-scharr}{Scharr()}},
\texttt{\href{http://opencv.willowgarage.com/documentation/cpp/image_filtering.html\#cv-laplacian}{Laplacian()}}
& Compute the first, second, third or mixed spatial image derivatives. \texttt{Laplacian()} computes $\Delta I = \frac{\partial ^ 2 I}{\partial x^2} + \frac{\partial ^ 2 I}{\partial y^2}$  \\

\texttt{\href{http://opencv.willowgarage.com/documentation/cpp/image_filtering.html\#cv-erode}{erode()}}, \texttt{\href{http://opencv.willowgarage.com/documentation/cpp/image_filtering.html\#cv-dilate}{dilate()}} & Erode or dilate the image \\

\end{tabular}

\begin{tabbing}
Exa\=mple. Filter image in-place with a 3x3 high-pass filter\\
\> (preserve negative responses by shifting the result by 128):\\
\texttt{filter2D(image, image, image.depth(), Mat(Mat\_<float>(3,3)}\\
\> \texttt{ << -1, -1, -1, -1, 9, -1, -1, -1, -1), Point(1,1), 128);}\\
\end{tabbing}

\subsection{Geometrical Transformations}

\begin{tabular}{@{}p{\the\MyLen}%
                @{}p{\linewidth-\the\MyLen}@{}}
\texttt{\href{http://opencv.willowgarage.com/documentation/cpp/geometric_image_transformations.html\#cv-resize}{resize()}} & Resize image \\

\texttt{\href{http://opencv.willowgarage.com/documentation/cpp/geometric_image_transformations.html\#cv-getrectsubpix}{getRectSubPix()}} & Extract an image patch with bilinear interpolation \\

\texttt{\href{http://opencv.willowgarage.com/documentation/cpp/geometric_image_transformations.html\#cv-warpaffine}{warpAffine()}} & Warp image using an affine transformation (rotation, scaling, shearing, reflection)\\

\texttt{\href{http://opencv.willowgarage.com/documentation/cpp/geometric_image_transformations.html\#cv-warpperspective}{warpPerspective()}} & Warp image using a perspective transformation\\

\texttt{\href{http://opencv.willowgarage.com/documentation/cpp/geometric_image_transformations.html\#cv-remap}{remap()}} & Generic image warping using the pre-computed maps\\

\texttt{\href{http://opencv.willowgarage.com/documentation/cpp/geometric_image_transformations.html\#cv-convertmaps}{convertMaps()}} & Optimize maps for a faster remap() execution\\

\end{tabular}

\begin{tabbing}
Example. Decimate image by factor of $\sqrt{2}$:\\
\texttt{Mat dst; resize(src, dst, Size(), 1./sqrt(2), 1./sqrt(2));}
\end{tabbing}

\subsection{Various Image Transformations}

\begin{tabular}{@{}p{\the\MyLen}%
                @{}p{\linewidth-\the\MyLen}@{}}

\texttt{\href{http://opencv.willowgarage.com/documentation/cpp/miscellaneous_image_transformations.html\#cvtColor}{cvtColor()}} & Convert image from one color space to another \\

\texttt{\href{http://opencv.willowgarage.com/documentation/cpp/miscellaneous_image_transformations.html\#threshold}{threshold()}}, \texttt{\href{http://opencv.willowgarage.com/documentation/cpp/miscellaneous_image_transformations.html\#adaptivethreshold}{adaptivethreshold()}} & Convert grayscale image to binary image using a fixed or a variable (location-dependent) threshold \\

\texttt{\href{http://opencv.willowgarage.com/documentation/cpp/miscellaneous_image_transformations.html\#floodfill}{floodFill()}} & Find a connected component starting from the specified seed point by region growing technique \\

\texttt{\href{http://opencv.willowgarage.com/documentation/cpp/miscellaneous_image_transformations.html\#floodfill}{integral()}} & Compute integral image, used further for to compute cumulative characteristics over rectangular regions in O(1) time \\

\texttt{\href{http://opencv.willowgarage.com/documentation/cpp/miscellaneous_image_transformations.html\#distancetransform}{distanceTransform()}},
 & build a distance map or a discrete Voronoi diagram from binary image. \\

\texttt{\href{http://opencv.willowgarage.com/documentation/cpp/miscellaneous_image_transformations.html\#floodfill}{watershed()}},
\texttt{\href{http://opencv.willowgarage.com/documentation/cpp/miscellaneous_image_transformations.html\#grabcut}{grabCut()}}
 & marker-based image segmentation algorithms. See
 See \texttt{\href{https://code.ros.org/svn/opencv/trunk/opencv/samples/c/watershed.cpp}{watershed.cpp}} and \texttt{\href{https://code.ros.org/svn/opencv/trunk/opencv/samples/c/grabcut.c}{grabcut.cpp}}
 samples.

\end{tabular}

\subsection{Histograms}

\begin{tabular}{@{}p{\the\MyLen}%
                @{}p{\linewidth-\the\MyLen}@{}}

\texttt{\href{http://opencv.willowgarage.com/documentation/cpp/histograms.html\#calchist}{calcHist()}} & Compute a histogram from one or more images \\

\texttt{\href{http://opencv.willowgarage.com/documentation/cpp/histograms.html\#calcbackproject}{calcBackProject()}} & Compute histogram back-projection (the posterior probability map) for the images \\

\texttt{\href{http://opencv.willowgarage.com/documentation/cpp/histograms.html\#equalizehist}{equalizeHist()}} & Normalize image brightness and contrast by equalizing the image histogram\\

\texttt{\href{http://opencv.willowgarage.com/documentation/cpp/histograms.html\#comparehist}{compareHist()}} & Compare two histograms\\

\end{tabular}

\begin{tabbing}
Example. Compute Hue-Saturation histogram of an image:\\
\texttt{Mat hsv, H; MatND tempH;}\\
\texttt{cvtColor(image, hsv, CV\_BGR2HSV);}\\
\texttt{int planes[]=\{0, 1\}, hsize[] = \{32, 32\};}\\
\texttt{calcHist(\&hsv, 1, planes, Mat(), tempH, 2, hsize, 0);}\\
\texttt{H = tempH;}
\end{tabbing}

\subsection{Contours}
See \texttt{\href{https://code.ros.org/svn/opencv/trunk/opencv/samples/cpp/contours.cpp}{contours.cpp}} and \texttt{\href{https://code.ros.org/svn/opencv/trunk/opencv/samples/c/squares.c}{squares.c}}
samples on what are the contours and how to use them.

\section{Data I/O}

XML/YAML storages are collections (possibly nested) of scalar values, structures and heterogeneous lists.

\begin{tabbing}
\textbf{Wr}\=\textbf{iting data to YAML (or XML)}\\
\texttt{// Type of the file is determined from the extension}\\
\texttt{FileStorage fs("test.yml", FileStorage::WRITE);}\\
\texttt{fs << "i" << 5 << "r" << 3.1 << "str" << "ABCDEFGH";}\\
\texttt{fs << "mtx" << Mat::eye(3,3,CV\_32F);}\\
\texttt{fs << "mylist" << "[" << CV\_PI << "1+1" <<}\\
\>\texttt{"\{:" << "month" << 12 << "day" << 31 << "year"}\\
\>\texttt{<< 1969 << "\}" << "]";}\\
\texttt{fs << "mystruct" << "\{" << "x" << 1 << "y" << 2 <<}\\
\>\texttt{"width" << 100 << "height" << 200 << "lbp" << "[:";}\\
\texttt{const uchar arr[] = \{0, 1, 1, 0, 1, 1, 0, 1\};}\\
\texttt{fs.writeRaw("u", arr, (int)(sizeof(arr)/sizeof(arr[0])));}\\
\texttt{fs << "]" << "\}";}
\end{tabbing}

\emph{Scalars (integers, floating-point numbers, text strings), matrices, STL vectors of scalars and some other types can be written to the file storages using \texttt{<<} operator}

\begin{tabbing}
\textbf{Re}\=\textbf{ading the data back}\\
\texttt{// Type of the file is determined from the content}\\
\texttt{FileStorage fs("test.yml", FileStorage::READ);}\\
\texttt{int i1 = (int)fs["i"]; double r1 = (double)fs["r"];}\\
\texttt{string str1 = (string)fs["str"];}\\

\texttt{Mat M; fs["mtx"] >> M;}\\

\texttt{FileNode tl = fs["mylist"];}\\
\texttt{CV\_Assert(tl.type() == FileNode::SEQ \&\& tl.size() == 3);}\\
\texttt{double tl0 = (double)tl[0]; string tl1 = (string)tl[1];}\\

\texttt{int m = (int)tl[2]["month"], d = (int)tl[2]["day"]};\\
\texttt{int year = (int)tl[2]["year"];}\\

\texttt{FileNode tm = fs["mystruct"];}\\

\texttt{Rect r; r.x = (int)tm["x"], r.y = (int)tm["y"];}\\
\texttt{r.width = (int)tm["width"], r.height = (int)tm["height"];}\\
 
\texttt{int lbp\_val = 0;}\\
\texttt{FileNodeIterator it = tm["lbp"].begin();}\\

\texttt{for(int k = 0; k < 8; k++, ++it)}\\
\>\texttt{lbp\_val |= ((int)*it) << k;}\\
\end{tabbing}

\emph{Scalars are read using the corresponding FileNode's cast operators. Matrices and some other types are read using \texttt{>>} operator. Lists can be read using FileNodeIterator's.}

\begin{tabbing}
\textbf{Wr}\=\textbf{iting and reading raster images}\\
\texttt{imwrite("myimage.jpg", image);}\\
\texttt{Mat image\_color\_copy = imread("myimage.jpg", 1);}\\
\texttt{Mat image\_grayscale\_copy = imread("myimage.jpg", 0);}\\
\end{tabbing}

\emph{The following formats are supported: \textbf{BMP (.bmp), JPEG (.jpg, .jpeg), TIFF (.tif, .tiff), PNG (.png), PBM/PGM/PPM (.p?m), Sun Raster (.sr), JPEG 2000 (.jp2)}. Every format supports 8-bit, 1- or 3-channel images. Some formats (PNG, JPEG 2000) support 16 bits per channel.}

\begin{tabbing}
\textbf{Re}\=\textbf{ading video from a file or from a camera}\\
\texttt{VideoCapture cap;}\\
\texttt{if(argc > 1) cap.open(string(argv[1])); else cap.open(0)};\\
\texttt{Mat frame; namedWindow("video", 1);}\\
\texttt{for(;;) \{}\\
\>\texttt{cap >> frame; if(!frame.data) break;}\\
\>\texttt{imshow("video", frame); if(waitKey(30) >= 0) break;}\\
\texttt{\} }
\end{tabbing}

\section{Simple GUI (highgui module)}

\begin{tabular}{@{}p{\the\MyLen}%
                @{}p{\linewidth-\the\MyLen}@{}}

\texttt{\href{http://opencv.willowgarage.com/documentation/cpp/user_interface.html\#cv-namedwindow}{namedWindow()}} & Create window with the specified name \\

\texttt{\href{http://opencv.willowgarage.com/documentation/cpp/user_interface.html\#cv-destroywindow}{destroyWindow()}} & Destroy the specified window \\

\texttt{\href{http://opencv.willowgarage.com/documentation/cpp/user_interface.html\#cv-imshow}{imshow()}} & Show image in the window \\

\texttt{\href{http://opencv.willowgarage.com/documentation/cpp/user_interface.html\#cv-waitKey}{waitKey()}} & Wait for a key press during the specified time interval (or forever). Process events while waiting. \emph{Do not forget to call this function several times a second in your code.} \\

\texttt{\href{http://opencv.willowgarage.com/documentation/cpp/user_interface.html\#cv-createTrackbar}{createTrackbar()}} & Add trackbar (slider) to the specified window \\

\texttt{\href{http://opencv.willowgarage.com/documentation/cpp/user_interface.html\#cv-setmousecallback}{setMouseCallback()}} & Set the callback on mouse clicks and movements in the specified window \\

\end{tabular}

See \texttt{\href{https://code.ros.org/svn/opencv/trunk/opencv/samples/c/camshiftdemo.c}{camshiftdemo.c}} and other \href{https://code.ros.org/svn/opencv/trunk/opencv/samples/}{OpenCV samples} on how to use the GUI functions.   

\section{Camera Calibration, Pose Estimation and Depth Estimation}

\begin{tabular}{@{}p{\the\MyLen}%
                @{}p{\linewidth-\the\MyLen}@{}}

\texttt{\href{http://opencv.willowgarage.com/documentation/cpp/camera_calibration_and_3d_reconstruction.html\#cv-calibratecamera}{calibrateCamera()}} & Calibrate monocular camera from multiple known projections of a calibration pattern feature points collected from several views. \\

\texttt{\href{http://opencv.willowgarage.com/documentation/cpp/camera_calibration_and_3d_reconstruction.html\#cv-findchessboardcorners}{findChessboardCorners()}} & \ \ \ \ \ \ Find feature points on the checkerboard calibration pattern with known geometry. \\

\texttt{\href{http://opencv.willowgarage.com/documentation/cpp/camera_calibration_and_3d_reconstruction.html\#cv-solvepnp}{solvePnP()}} & Find the object pose from the known projections of its feature points. \\

\texttt{\href{http://opencv.willowgarage.com/documentation/cpp/camera_calibration_and_3d_reconstruction.html\#cv-stereocalibrate}{stereoCalibrate()}} & Calibrate stereo camera using several stereo views of a calibration pattern. \\

\texttt{\href{http://opencv.willowgarage.com/documentation/cpp/camera_calibration_and_3d_reconstruction.html\#cv-stereorectify}{stereoRectify()}} & Compute the rectification transforms for a stereo camera and the visible area on the rectified images. Camera must be calibrated first using stereoCalibrate().\\

\texttt{\href{http://opencv.willowgarage.com/documentation/cpp/camera_calibration_and_3d_reconstruction.html\#cv-initundistortrectifymap}{initUndistortRectifyMap()}} & \ \ \ \ \ \ Compute rectification map (for \texttt{remap()}) for each head of a stereo camera. Must be called twice, for each head, after \texttt{stereoRectify()}.\\

\texttt{\href{http://opencv.willowgarage.com/documentation/cpp/camera_calibration_and_3d_reconstruction.html\#cv-StereoBM}{StereoBM}}, \texttt{\href{http://opencv.willowgarage.com/documentation/cpp/camera_calibration_and_3d_reconstruction.html\#cv-StereoSGBM}{StereoSGBM}} & The two primary stereo correspondence algorithms in OpenCV. They work on the rectified images.\\

\texttt{\href{http://opencv.willowgarage.com/documentation/cpp/camera_calibration_and_3d_reconstruction.html\#cv-reprojectimageto3d}{reprojectImageTo3D()}} & Convert the disparity map, computed by \texttt{StereoBM::operator ()} or \texttt{StereoSGBM::operator ()} to the 3D point cloud.\\

\end{tabular}

To calibrate a camera, you can use \texttt{\href{https://code.ros.org/svn/opencv/trunk/opencv/samples/c/calibration.cpp}{calibration.cpp}} or
\texttt{\href{https://code.ros.org/svn/opencv/trunk/opencv/samples/c/stereo\_calib.cpp}{stereo\_calib.cpp}} samples.
To run stereo correspondence and optionally get the point clouds, you can use
\texttt{\href{https://code.ros.org/svn/opencv/trunk/opencv/samples/c/stereo\_match.cpp}{stereo\_match.cpp}} sample.

\section{Object Detection}

\begin{tabular}{@{}p{\the\MyLen}%
                @{}p{\linewidth-\the\MyLen}@{}}
                \texttt{\href{http://opencv.willowgarage.com/documentation/cpp/object_detection.html\#matchTemplate}{matchTemplate}} & Primitive Viola's Cascade of Boosted classifiers using Haar or LBP features. Detects objects by sliding a window and running the cascade on them. Suits for detecting faces, facial features and some other objects without diverse textures.\\

\texttt{\href{http://opencv.willowgarage.com/documentation/cpp/object_detection.html\#CascadeClassifier}{CascadeClassifier}} & Viola's Cascade of Boosted classifiers using Haar or LBP features. Detects objects by sliding a window and running the cascade on them. Suits for detecting faces, facial features and some other objects without diverse textures. See \texttt{\href{https://code.ros.org/svn/opencv/trunk/opencv/samples/c/facedetect.cpp}{facedetect.cpp}}\\

\texttt{\href{http://opencv.willowgarage.com/documentation/cpp/object_detection.html\#HOGDescriptor}{HOGDescriptor}} & N. Dalal's object detector using Histogram-of-Oriented-Gradients (HOG) features. Detects objects by sliding a window and running SVM classifier on them. Suits for detecting people, cars and other objects with well-defined silhouettes. See \texttt{\href{https://code.ros.org/svn/opencv/trunk/opencv/samples/c/peopledetect.cpp}{peopledetect.cpp}}\\

\end{tabular}

%    
%    feature detection:
%        features2d toolbox
%    
%    object detection:
%        using a classifier running on a sliding window: cascadeclassifier + hog.
%        using salient point features: features2d -> matching
%        
%    statistical data processing:
%        clustering (k-means),
%        classification + regression (SVM, boosting, k-nearest),
%        compressing data (PCA)

\end{multicols}
\end{document}
